\documentclass[12pt,journal,compsoc]{D:/Магистратура/English/bare_conf/IEEEtran}
\newcommand\MYhyperrefoptions{
pdfpagemode={UseOutlines},plainpages=false,pdfpagelabels=true,
colorlinks=true,linkcolor={black},citecolor={black},urlcolor={black},
pdftitle={Working with attached to repair documents in the system of automatic registration of repairs},
pdfsubject={Typesetting},
pdfauthor={Arseny Zorin}}

\begin{document}
\title{Working with attached to repair documents in the system of automatic registration of repairs}

\author{Arseny~Zorin}
\maketitle

\IEEEpeerreviewmaketitle



\section{Introduction}
\IEEEPARstart{S}ince the appearance of computers, progress isn't standing still. Technology has enormously improved in last decade. Among trends that appear one can find an ambient wireless Internet access to printing prosthesis on 3D printers or cloud storages. The latter became storage de-facto standard for past few years. It is used not only for private, but also for corporates document storing and data archiving.

The object of research is the authorized service center (ASC) that provides warranty and paid repair of household appliances. There are four branches situated in different parts of the city that take equipment for repair. Each repair operation performed by ASC produces an number of attached documents:

- A photo of schild from back of equipment. This file is stored in a common folder for schilds' photos of this manufacturer. File name equals to the ticket number.

- A scanned warranty ticket. This file is stored in a common folder for scanned documents of this manufacturer. File name equals to the ticket number.

- An outbound repair receipt. Produced in case the repair was carried out at the client's residence. This file is stored in a common folder of outbound repairs for this manufacturer. File name equals to the ticket number.

- Various acts provided to a client including a technical condition act, a non-repairable act, a replacing device act.

All document folders are stored at branches offices and at the head office server. Data is transmitted between branches and the head office through FTP protocol.

Thi work was motivated by a need for modernization of tools for documents storage and transmition. The software programming language was C\# instead of VBA. For scanning documents and making photo of them new libraries will be used. All documents storage were moved to the cloud. Data will be stored and treated at the distributed servers in network. While using cloud storages, users have a possibility to access the data from different places and from different devices which have an Internet access. Also cloud storage can prevent losing information in case of failure of a local storage.

\section{The study of the subject}

\subsection{Range of development tools}

\subsection{Interaction with webcam}

\subsection{Interaction with scanner}

\subsection{Working with documents}

\subsection{Cloud storage}

\subsubsection{OneDrive}

\subsubsection{Dropbox}

\subsubsection{Yandex.Disk}

\subsubsection{Google Drive}

\section{Practical implementation}

\subsection{Working with AForge.NET}

\subsection{Working with TWAIN library}

\subsection{Working with Microsoft.Office.Interop}

\subsubsection{Working with Microsoft.Office.Interop.Word}

\subsubsection{Working with Microsoft.Office.Interop.Excel}

\subsection{Working with Yandex.Disk}

\section{Conclusion}


\end{document}
