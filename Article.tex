\documentclass[12pt,journal,compsoc]{D:/Магистратура/English/bare_conf/IEEEtran}
\newcommand\MYhyperrefoptions{
pdfpagemode={UseOutlines},plainpages=false,pdfpagelabels=true,
colorlinks=true,linkcolor={black},citecolor={black},urlcolor={black},
pdftitle={Working with attached to repair documents in the system of automatic registration of repairs},
pdfsubject={Typesetting},
pdfauthor={Arseny Zorin}}

\begin{document}
\title{Working with attached to repair documents in the system of automatic registration of repairs}

\author{Arseny~Zorin}
\maketitle

\IEEEpeerreviewmaketitle



\section{Introduction}
\IEEEPARstart{S}ince the appearance of computers, progress isn't standing still. Technology have enormously improved in last decade. From wireless access to the Internet to printing prosthesis on 3D printers. Cloud storages became one of the most spread technology for past few years. It's used not only in private using, but in different companies for storing documents and archives of data too.

The object of work is the real authorized service center that provides warranty and paid repair of household appliances. There are four branches that take equipments for repair and situated in different parts of the city. Each repair produced by ASC has different attached documents:

  - Photo of schild from back of equipment. This file stores in common folder for schilds' photos of this manufacturer. File name equal to the ticket number.

- Scanned warranty ticket. This file stores in common folder for scanned documents of this manufacturer. File name equals to the ticket number.

- Receipt of outbound repair. If repair was carried out on a client home. File stores at common folder of outbound repairs for this manufacturer. File name equals to the ticket number.

- Various acts written to a client - act of the technical condition, non-repairable act, act of replacing device.

All folders with documents are storing on branches and on the server of head office. Between branches and the head office data is transmitted through FTP protocol.

The relevance of the work lies in the implementation of the tasks weth more modern means than used in authorized service center. In case of programming language it will be C\# instead of VBA. For scanning documents and making photo of them new libraries will be used. In case of storage in this work local storage will be replaced with cloud storage. Data will be storing and treating on distributed servers in network. While using cloud storages, users have possibility to access to data from different places and from different devices which have an access to the Internet. Also cloud storage can prevent from losing information in case of failure of local storage.

\section{The study of the subject}

\subsection{Range of development tools}

\subsection{Interaction with webcam}

\subsection{Interaction with scanner}

\subsection{Working with documents}

\subsection{Cloud storage}

\subsubsection{OneDrive}

\subsubsection{Dropbox}

\subsubsection{Yandex.Disk}

\subsubsection{Google Drive}

\section{Practical implementation}

\subsection{Working with AForge.NET}

\subsection{Working with TWAIN library}

\subsection{Working with Microsoft.Office.Interop}

\subsubsection{Working with Microsoft.Office.Interop.Word}

\subsubsection{Working with Microsoft.Office.Interop.Excel}

\subsection{Working with Yandex.Disk}

\section{Conclusion}


\end{document}